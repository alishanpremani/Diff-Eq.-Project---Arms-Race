% This is a LaTeX source file with extensive commentary for learning LaTeX.
% It is a modified version of a file due to Dr. Kevin P. Lee, now at Chicago's Simeon Career Academy High School.

% You can used this file as a LaTeX template.
%Copy this file and make any adaptations which you need.
%Consult a LaTeX guide for further information about TeX and LaTeX.

% The Very Basics:

%	1. Anything on a line after the character ''%'' appears in red and is ignored 
%		by LaTeX. Thus you can delete any line beginning with ''%" and nothing about the output
%		will change (although these comments are often very helpful).

%	2. The backslash "\" precedes commands.

%	3. Command arguments are enclosed by braces "{}".

%	4. All LaTex source files (this is a source file; it has the suffix .tex) begin with 
%		the line \documentclass{"class"}, where "class" specifies a particular
%		type of formatting.

%	5. There are two main parts of a source file:
%		a) Everything after \documentclass but before \begin{document} is
%			called the "preamble" -- it defines properties of your document.

%		b) The text of your document occurs between the commands
%			\begin{document} and the very last command \end{document}.


\documentclass{article}	% This tells LaTeX to format the content into the style of an article.
                                        
%\documentclass[11pt]{article}		% You can include some options within the 
							% square brackets. These options will affect 
                                        			% the formatting of the entire document. The
                                      			% option here selects for a type size of
                                        			% 11 points. This line is currently ignored
                                       			% by LaTeX. To use this option, add a "%" at
							% the beginning of the other \documentclass
							% command and delete the ''%'' at the
							% beginning of this line. Similar instructions
							% apply to each of the following lines.

%\documentclass[12pt]{article}          % You can also select a type size of 12 points.

%\documentclass[legalpaper]{article}	% This causes the output to be formatted
								% onto a legal-sized document.

%\documentclass[landscape]{article}	% This causes the output to be formatted
								% ''sideways''.

%\documentclass[twocolumn]{article}	% You can specify two columns on each page.

%\documentclass[leqno]{article}	% The ''leqno'' option specifies that equation 
                                        			% numbers will be placed on the left instead of 
                                        			% the right.

%\documentclass[fleqn]{article}         % Instead of centering displayed formulas,
							% formulas will be aligned on the left.

%\documentclass{report}			% This line will cause the content to be formatted 
							% into a report. The report class is better for 
							% longer documents.
                                        
%\documentclass{book}			% The book class is best for very long documents.

% You can combine various document options and classes. For instance, suppose that
% you want a book written in 12 point typeface, two columns, and with equations
% aligned on the left. Then you should specify:

%\documentclass[12pt,twocolumn,fleqn]{book}

%%%%%%%%%%%%%%%%%%%%%%%%%%%%%%%%%%%%%%%%%%%%%%%%%%%%%%%%%%%%%%%%%%%%%%%%%%%%%%%%

% Now we begin the PREAMBLE:

% The \usepackage command will read additional files which will either affect the
% formatting of the entire document or allow you to use additional commands. 
% Some files are standard. Others can be downloaded from the web. Typically,
% these extra files will end in the suffix ''.sty''. So the command
% ''\usepackage{example}'' will cause LaTeX to read the file ''example.sty''.

% These next several \usepackage commands will cause LaTeX to use different fonts.
% Notice that all of these lines will be ignored by LaTeX unless you delete one of
% the  ''%'' characters at the beginning of a line.

%\usepackage{times}	% The package ''times'' will cause LaTeX to use the
					% Times font throughout most of your document.

%\usepackage{palatino}		% Use Palatino.
%\usepackage{bookman}		% Use Bookman.
%\usepackage{newcent}		% Use New Century Schoolbook.
%\usepackage{garamond}	% Use Garamond.

% These next few commands will allow you to use some additional mathematical
% typesetting commands.

\usepackage{amsmath}		% The ''amsmath'' package will allow you to use useful 
						% commands contained within the AMS-LaTeX package.
						% (AMS is the American Mathematical Society.)
					
\usepackage{amssymb}		% The ''amssymb'' package will allow you to use 
						% additional mathematical symbols.
						
\usepackage{amsthm}		% The ''amsthm'' package contains commands
						% for formatting mathematical theorems and 
						% statements.

% These next packages allow you to include graphics within a LaTeX document.
% Please read the available documentation before trying to use these packages.
% They will not be loaded due to the ''%'' characters at the beginning of the lines.
%  To use these packages, delete the ''%'' characters.

%\usepackage{graphicx}

%\usepackage{setspace}		% The ``setspace'' package contains new 
						% commands that will facilitate doublespacing 
						% a document. Make sure that you have the file 
 						% ''setspace.sty'' contained in the same folder as 
						% the source file.
                                        
%\doublespacing			% The ''\doublespacing'' command is available only if
						% ''\usepackage{setspace}'' is in the document. This 
						% causes the document to be double-spaced (surprise!).
                                        
%\onehalfspacing			% Alternatively, one can use ''\onehalfspacing'' with 
						% the ''setspace'' package.

% If you do not want to use the ''setspace'' package, you could use the command
% ''\baselinestretch{2}'' or ''\baselinestretch{1.5}'' in your document. In some
% situations though, this produces bad results.

%\usepackage[top=1in,bottom=1in,left=1.25in,right=1in]{geometry}	   % The ``geometry" package
					% allows you to specify the margins of the document. For example, the
					% options indicated in the square brackets set all margins except the left 
					% to 1" and the left margin at 1.25".

%%%%%%%%%%%%%%%%%%%%%%%%%%%%%%%%%%%%%%%%%%%%%%%%%%%%%%%%%%%%%%%%%%%%%%%%%%%%%%%%

% Now we begin the BODY OF YOUR DOCUMENT:

\begin{document}	% All LaTeX files contain the command ''\begin{document}''. 
				% There is a corresponding ''\end{document}'' command
				% at the end.

% Next we'll give basic titling information to LaTeX:

\title{A Sample \LaTeX ~Document}		% The title of the document.
								% The ~ symbol forces a space; try removing it to see 
								% what happens.

\author{Alishan}					% The author

\date{\today}				% This date-stamps your paper automatically. If you want to 
						% specify a fixed date, you may do so between the braces.
						% If you don't want a date to appear, type ''\date{}''.

% Creating the title section:

\maketitle		% This command creates the appropriate header. It may alternatively
			% create a title page. This will depend on whether you have selected  
 			% ''report'', ''book", or ''article'' in the ''\documentclass'' command above.

\section{Some Typesetting Commands}		% The ''\section'' command tells LaTeX
									% to begin a new section. There will be
									% a header titled "Some Typesetting 
									% Commands". There are also the 
									% commands ''\subsection'' and 
									% ''\subsubsection''. These commands 
									% are used in a similar manner.
									
%\section*{Some Typesetting Commands}	% Alternatively, there is the ''\section*''
									% command. This is just like the ''\section''
									% command, but LaTeX will not create 
									% any number for the section. There are
									% also * versions of the ''\subsection''
									 % and ''\subsubsection'' commands.

See the \emph{Introduction to \LaTeX} for further information on entering text into \LaTeX.
I would now like to quote Lewis Carroll\footnote{From \emph{Through
the Looking Glass}.}:
\begin{quote}
``A slow sort of country!'' said the Queen.
``Now, \emph{here}, you see, it takes all the running \emph{you} can do, to keep in the same place.
If you want to get somewhere else, you must run twice as fast as that!''
\end{quote}

% Hopefully it is clear what the ''\footnote'' command does.
% One wonderful thing about LaTeX is that it takes care of the numbering and placement of footnotes automatically.
% If you add a footnote later on, LaTeX will renumber all of them.

% Pay close attention to how you tell LaTeX to include beginning quotation marks: use `` rather than ".
%(The ` symbol is found in the upper left corner of the keyboard.)

% What follows is a new paragraph.
% LaTeX knows this because we have left at least one blank line between it and the preceding paragraph.
% LaTeX will thus indent the start of the following text:

Now I would like to demonstrate some different type styles:
\textrm{Roman}, \textit{Italic}, \textbf{Bold}, \textsl{Slanted}, \textsc{Small Caps},
\texttt{Typewriter}, and \textsf{San Serif}.

% The ''\emph'' command is not an italics command.
% It instead stands for ''emphasis''. 
% This will italicize text in most typical situations, but it can do other things depending on context.
% The commands `''\textrm'', ''\textit'', etc. are more explicit ways of changing type styles.

%Notice in the above sentence that even though the source file has a line break after the colon, the LaTeXed document does not.
%To start a new line in the document without starting a new paragraph, put ''\\'' at the end of the line.


Here is an example \\
of a new line without starting a new paragraph.



And now a list:
\begin{itemize}
	\item Canada
	\item United States
	\item Mexico
\end{itemize}

And another list:
\begin{enumerate}
	\item United Kingdom
	\item China
	\item Russia
	\item France
\end{enumerate}

% The command ''itemize'' creates a list with bullets.
%The command ''enumerate'' creates a numbered list.
%Notice how LaTeX automatically does the numbering.
%You can embed lists within lists.
%The indentation above is not necessary, but it makes the lists easier to read in the source file.

Finally, some examples of special characters:
\begin{quote}
We drove down La Ca\~{n}ada Drive to watch Wagnerian opera.
The role of Br\"{u}nnhilde was played by Ren\'{e}e Gossens.
Later we went to a restaurant called ``La Meurtri\`{e}re'' and were served r\^{o}ti by a rude gar\c{c}on.
\end{quote}

\section{Tables}

Here is a one-inch space followed by a sample table:

\vspace{1.0in}		% The ''\vspace'' command causes LaTeX to leave some vertical 
				% space (in this case, 1 inch of space). You can use this command
				% to leave space for illustrations or diagrams which are to be  
				% pasted into the document manually, or simply to control spacing.
				% Use this command sparingly. Let LaTeX do the spacing work for you.

\begin{tabular}{|l|r|}	% The ''\begin{tabular}'' command begins a table. There is a 
				% corresponding ''\end{tabular}'' command after the table. 
				% The ''{|l|r|}'' after the ''\begin{tabular}'' command is, unfortunately,
				% difficult to decipher here. It consists of the letters ''l'' (el) and ''r''
				% and vertical bars in this order: bar, el, bar, r, bar. This indicates
				% that the table will have two columns, the first one flushed left,
				% the second one flushed right. (The letter ''c'' would indicate a 
				% centered column.) The bars indicates that there will be vertical
				% lines preceding and following each column.
                                            
\hline	% The command ''\hline'' indicates a horizontal line in the table.
                                            
\textbf{Original} & \textbf{Pig Latin} \\		% The start of each column after the first is 
								% indicated by a ''&'' character. The end of 
								% the row is indicated by typing ''\\''.
\hline
\hline
Cat & Atcay \\
Dog & Ogday \\
Trash & Ashtray \\
\hline
\end{tabular}		% The end of the table.

\section{And Now Some Mathematics}

Mathematical expressions, like $x^{2} + 3$, are easy to create in \LaTeX.
It is also easy to have a formula written on a separate line, like
\[x^{2} + 3 = 0.\]

% Mathematical expressions must be contained in a pair of single dollar signs ''$ ... $'' or ''\[ ... \]''.
%A pair of single dollar signs keeps the math in the same line as the text.
%To place the math centered, on its own line with larger symbols, surround it by ''\[''' and "\]".
%This is called display style.

\noindent Numbered equations are easy too!
\begin{equation}
x^{5} - 9x^{4} = \sin 3.
\end{equation}
You can also refer to previous equations.
The formula
\begin{equation}
x = \frac{-b \pm \sqrt{b^{2} - 4ac}}{2a} \label{yourlabel}
\end{equation}
is the quadratic formula.
You can use Equation \ref{yourlabel} to solve quadratic equations.

% Notice that you can also place equations and formulas between the commands 
% ''\begin{equation}'' and ''\end{equation}''. No dollar signs are necessary in this case. 

% You can also create a label for the equation using the command ''\label''. You can 
% refer back to the equation using the ''\ref'' command. Notice how LaTeX 
% automatically takes care of the numbering. When you use the ''\ref'' command, you 
% will need to run LaTeX on the source file twice. This is necessary so that LaTeX 
% can keep track of the numbering. (Note that LaTeX will complain a little bit the first 
% time you run the source file through. This is not serious.)

Here is a formula with some simple mathematical operations:
\[5 \times 3 + 4 \div 2 = 2 \cdot 10 - 3 < 100.\]
Another simple formula:
\[x^{2} \geq 0.\]			% The command ''\geq'' is ''greater than or equal'' and 
					% ''\leq'' is ''less than or equal''.				
This formula is true for $x \in \mathbb{R}$.	% The command ''\mathbb'' specifies 
									% a certain type of math font. There are 
									% also the ''\mathbf'' (bold) and the 
									% ''\mathcal'' (calligraphic) commands, 
									% among others, for specifying font styles 
									% in math formulas.
One more simple formula:
\[0 \neq \frac{\pi}{2}.\]
Some simple trigonometry:
\[\sin(2x) = 2 \sin x \cos x.\]
Some simple logarithms:
\[\log(x) = \frac{\ln x}{\ln 10}.\]
And now some Greek letters:
\[\alpha + \beta = \gamma.\]
A formula involving set notation:
\[A \cup (B \cap C) = (A \cup B) \cap (A \cup C).\]
The derivative:
\[\lim_{h \rightarrow 0}\frac{f(x + h) - f(x)}{h} = f'(x).\]
The Riemann sum
\[\sum_{i = 1}^{n}f(x_{i})\Delta x\]
approximates the integral
\[\int_{a}^{b}f(x)\;dx.\]			% The characters ``\;'' just place some additional space 
						% in the formula.
Finally, an infinite series:
\[1, \frac{1}{2}, \frac{1}{4}, \dots.\]

\end{document}                          % This command indicates the end of the file.
