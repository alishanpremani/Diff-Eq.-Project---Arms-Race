% This is a LaTeX source file with extensive commentary for learning LaTeX.
% It is a modified version of a file due to Dr. Kevin P. Lee, now at Chicago's Simeon Career Academy High School.

% You can used this file as a LaTeX template.
%Copy this file and make any adaptations which you need.
%Consult a LaTeX guide for further information about TeX and LaTeX.

% The Very Basics:

%	1. Anything on a line after the character ''%'' appears in red and is ignored 
%		by LaTeX. Thus you can delete any line beginning with ''%" and nothing about the output
%		will change (although these comments are often very helpful).

%	2. The backslash "\" precedes commands.

%	3. Command arguments are enclosed by braces "{}".

%	4. All LaTex source files (this is a source file; it has the suffix .tex) begin with 
%		the line \documentclass{"class"}, where "class" specifies a particular
%		type of formatting.

%	5. There are two main parts of a source file:
%		a) Everything after \documentclass but before \begin{document} is
%			called the "preamble" -- it defines properties of your document.

%		b) The text of your document occurs between the commands
%			\begin{document} and the very last command \end{document}.


%\documentclass{article}	% This tells LaTeX to format the content into the style of an article.
                                        
%\documentclass[11pt]{article}		% You can include some options within the 
							% square brackets. These options will affect 
                                        			% the formatting of the entire document. The
                                      			% option here selects for a type size of
                                        			% 11 points. This line is currently ignored
                                       			% by LaTeX. To use this option, add a "%" at
							% the beginning of the other \documentclass
							% command and delete the ''%'' at the
							% beginning of this line. Similar instructions
							% apply to each of the following lines.

\documentclass[12pt]{article}          % You can also select a type size of 12 points.

%\documentclass[legalpaper]{article}	% This causes the output to be formatted
								% onto a legal-sized document.

%\documentclass[landscape]{article}	% This causes the output to be formatted
								% ''sideways''.

%\documentclass[twocolumn]{article}	% You can specify two columns on each page.

%\documentclass[leqno]{article}	% The ''leqno'' option specifies that equation 
                                        			% numbers will be placed on the left instead of 
                                        			% the right.

%\documentclass[fleqn]{article}         % Instead of centering displayed formulas,
							% formulas will be aligned on the left.

%\documentclass{report}			% This line will cause the content to be formatted 
							% into a report. The report class is better for 
							% longer documents.
                                        
%\documentclass{book}			% The book class is best for very long documents.

% You can combine various document options and classes. For instance, suppose that
% you want a book written in 12 point typeface, two columns, and with equations
% aligned on the left. Then you should specify:

%\documentclass[12pt,twocolumn,fleqn]{book}

%%%%%%%%%%%%%%%%%%%%%%%%%%%%%%%%%%%%%%%%%%%%%%%%%%%%%%%%%%%%%%%%%%%%%%%%%%%%%%%%

% Now we begin the PREAMBLE:

% The \usepackage command will read additional files which will either affect the
% formatting of the entire document or allow you to use additional commands. 
% Some files are standard. Others can be downloaded from the web. Typically,
% these extra files will end in the suffix ''.sty''. So the command
% ''\usepackage{example}'' will cause LaTeX to read the file ''example.sty''.

% These next several \usepackage commands will cause LaTeX to use different fonts.
% Notice that all of these lines will be ignored by LaTeX unless you delete one of
% the  ''%'' characters at the beginning of a line.

%\usepackage{times}	% The package ''times'' will cause LaTeX to use the
					% Times font throughout most of your document.

%\usepackage{palatino}		% Use Palatino.
%\usepackage{bookman}		% Use Bookman.
%\usepackage{newcent}		% Use New Century Schoolbook.
%\usepackage{garamond}	% Use Garamond.

% These next few commands will allow you to use some additional mathematical
% typesetting commands.

\usepackage{amsmath}		% The ''amsmath'' package will allow you to use useful 
						% commands contained within the AMS-LaTeX package.
						% (AMS is the American Mathematical Society.)
					
\usepackage{amssymb}		% The ''amssymb'' package will allow you to use 
						% additional mathematical symbols.
						
\usepackage{amsthm}		% The ''amsthm'' package contains commands
						% for formatting mathematical theorems and 
						% statements.

% These next packages allow you to include graphics within a LaTeX document.
% Please read the available documentation before trying to use these packages.
% They will not be loaded due to the ''%'' characters at the beginning of the lines.
%  To use these packages, delete the ''%'' characters.

\usepackage{graphicx}

\usepackage{setspace}		% The ``setspace'' package contains new 
						% commands that will facilitate doublespacing 
						% a document. Make sure that you have the file 
 						% ''setspace.sty'' contained in the same folder as 
						% the source file.
                                        
\doublespacing			% The ''\doublespacing'' command is available only if
						% ''\usepackage{setspace}'' is in the document. This 
						% causes the document to be double-spaced (surprise!).
                                        
%\onehalfspacing			% Alternatively, one can use ''\onehalfspacing'' with 
						% the ''setspace'' package.

% If you do not want to use the ''setspace'' package, you could use the command
% ''\baselinestretch{2}'' or ''\baselinestretch{1.5}'' in your document. In some
% situations though, this produces bad results.

\usepackage[top=1in,bottom=1in,left=1in,right=1in]{geometry}	   % The ``geometry" package
					% allows you to specify the margins of the document. For example, the
					% options indicated in the square brackets set all margins except the left 
					% to 1" and the left margin at 1.25".

%%%%%%%%%%%%%%%%%%%%%%%%%%%%%%%%%%%%%%%%%%%%%%%%%%%%%%%%%%%%%%%%%%%%%%%%%%%%%%%%

% Now we begin the BODY OF YOUR DOCUMENT:


\begin{document}	% All LaTeX files contain the command ''\begin{document}''. 
				% There is a corresponding ''\end{document}'' command
				% at the end.

% Next we'll give basic titling information to LaTeX:

\title{MODELING THE ARMS RACE}		% The title of the document.
								% The ~ symbol forces a space; try removing it to see 
								% what happens.


\author{Jonathan Ledesma, Alishan Premani, Khoa Nguyen}					% The author


\date{\today}				% This date-stamps your paper automatically. If you want to 
						% specify a fixed date, you may do so between the braces.
						% If you don't want a date to appear, type ''\date{}''.

% Creating the title section:

\maketitle		% This command creates the appropriate header. It may alternatively
			% create a title page. This will depend on whether you have selected  
 			% ''report'', ''book", or ''article'' in the ''\documentclass'' command above.
			
%pull, add commit, push

\begin{center}
{\large \bfseries Introduction} % Major section
\end{center}

Every country is concerned about its national security. Maintaining an inventory of weaponry is one of the priorities in defense, but how a country does it depends on not only its own inventory, but also on other factors, such as technological advances, other countries's weaponry, and the tension among them. In this paper we seek to create a simple differential equations model to investigate the change in weaponry of two countries in relation to one another, and to improve our model by factoring another variable into our equations. Then we apply our model to the Cold War between the United States and the USSR where there was an arm race between the two countries. Due to limited time and resources, as well as techniques and skills in solving differential equations, our model has a lot of room for improvement, considering the number of variables not present in our equations.

\newpage

% Khoa
\section{Assumptions}
We define the following variables:
\begin{itemize}
\item $x(t) \ge 0$ is the amount of weaponry that country X has at time $t$,
\item $y(t) \ge 0$ is the amount of weaponry that country Y has at time $t$,
\item $a \ge 0$ is the coefficient of weapon production of country X,
\item $b \ge 0$ is the coefficient of weapon production of country Y.
\end{itemize}
We restrict this model to a specific time interval in which $a$ and $b$ stay constant, even though in a real-world situation, the production rates might fluctuate.

%JONATHAN LEDESMA
\section{Simple Model}		

%Alishan
\section{A More Realistic Model}
Now we consider the situation when both countries decide to update their inventory of weaponry due to new technological advances and old weapons being expired. 
Therefore, we define additional parameters in the differential equations:
\begin{align}
\frac{dx}{dt} & = ay - cx \\
\frac{dy}{dt} & = bx - dy
\end{align}	
where $c \ge 0$ and $d \ge 0$ are constant coefficients of the destruction of old weapons in countries X and Y, respectively. \\
We can solve this system of differential equations by constructing a coefficient matrix:
\[
M=
  \begin{bmatrix}
    -c & a\\ 
    b & d
  \end{bmatrix}.
\]
Our characteristic polynomial is $\lambda^2 + (d+c)\lambda + cd - ab$. Computing the eigenvalues and eigenvectors yield
$$\lambda_1 = -\frac{1}{2}d - \frac{1}{2}c + \frac{1}{2}\sqrt{4ab -2cd + c^2 + d^2}; \quad \vec{v}_1 = 
\begin{bmatrix}
	\frac{a}{\lambda_1}\\
	1
\end{bmatrix} $$

$$\lambda_2 = -\frac{1}{2}d - \frac{1}{2}c - \frac{1}{2}\sqrt{4ab -2cd + c^2 + d^2}; \quad \vec{v}_2 = 
\begin{bmatrix}
	\frac{a}{\lambda_2}\\
	1
\end{bmatrix} $$

\section{Case Study: Cold War}
We consider the arm race between the United States and the USSR during the Cold War. Due to limited time and resources, we define ``weaponry" in this case to only include warheads. We also exclude other external factors such as internal political conflicts in each country, current wars at the time, and treaties.


\section{Reference}
%https://lupucezar.files.wordpress.com/2011/02/yuan.pdf

\end{document}                          % This command indicates the end of the file.
